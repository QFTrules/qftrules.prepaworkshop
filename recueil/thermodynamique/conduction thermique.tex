% !TEX root = /home/eb/.vscode/qftrules.prepaworkshop/tmp/Exercice.tex
%%%%%%%%%%%%%%%%%%%%%%%%%%%%%%%%%%%
\begin{exo}[1][python]{Méthode d'Euler et équation de diffusion}

  \paragraph{Introduction}
    On s'intéresse ici à une barre métallique cylindrique dont la surface latérale est calorifugée. Les deux extrémités gauche et droite de la barre sont en contact avec de la glace fondante de température $T_g = \SI{0}{\degree C}$, qui joue le rôle de thermostat.  L’objectif de cet exercice est de résoudre numériquement l’équation de la diffusion thermique dans cette situation simple. On appelle désormais $L=\SI{1}{m}$ la longueur de la barre. On rappelle que l'équation de diffusion thermique s'écrit
    $$
    \boxed{
    \frac{\partial T}{\partial t} = \kappa \frac{\partial^2 T}{\partial x^2},
    }
    $$
    où $\kappa$ désigne la diffusivité thermique. On considérera un matériau de diffusivité $\kappa=\SI{e-4}{m^2.s^{-1}}$.
    
    
    Nous allons dans la suite déterminer la température $T(x,t)$, en tout point $x$ de la barre,
     et à tout instant $t$ de l'expérience. Pour cela, nous allons découper la barre en tronçons de longueur $\mathrm{d} x=\SI{1}{cm}$.
    L'étude sera menée sur une durée de $\Delta t = \SI{30}{min}$, avec un pas temporel de $\mathrm{d} t=\SI{0,4}{s}$. La fonction $T(x,t)$ sera donc approximée par un tableau \texttt{T[i,j]} de telle sorte que :
    $$
    \boxed{
      \texttt{T[i,j]} = T(x=j dx,t=i dt).
    }$$
    
    Autrement dit, chaque ligne \texttt{i} du tableau correspond à l'instant $t=i\mathrm{d}t$ de l'expérience (la ligne \texttt{i=0} correspondant à l'instant initial), et chaque colonne \texttt{j} du tableau représente la position $x=j\mathrm{d}x$ (la colonne \texttt{j=0} correspondant au bord gauche).
    
    Le tableau utilisé est de type \texttt{numpy.array}. On aura de plus besoin du module \texttt{matplotlib.pyplot} pour effectuer les représentations graphiques du champ de température. On importe pour cela les modules suivants. 
\begin{minted}{python}
import numpy as np
import matplotlib.pyplot as plt
\end{minted}
    
    \paragraph{Initialisation}
    \begin{questions}
      \item	
    
      On nommera désormais les variables entières \texttt{Nt} et \texttt{Nx} correspondant respectivement au nombre de lignes  et de colonnes du tableau \texttt{T}.
      
      Compléter les lignes suivantes pour définir \texttt{Nx} et \texttt{Nt} sur Python, et créer le tableau \texttt{T} en le remplissant de zéros pour l'instant. Créer également les tableaux \texttt{x} et \texttt{t} correspondant respectivement aux positions $x$ et aux instants $t$ pour lesquels la température est évaluée.
\begin{minted}{python}
dx = 1e-2
dt = 4e-1
L  = 1
Deltat = 30*60
\end{minted}
    \solution*{
\begin{minted}{python}
Nx = int(L/dx) + 1
Nt = int(Deltat/dt) + 1

T = np.zeros((Nt,Nx))

x = np.array([j*dx for j in range(Nx)])
t = np.array([i*dt for i in range(Nt)])
\end{minted}
    }
    \item On suppose que le profil de température initial a pour expression 
    $$
    T(x,t=0) = T_0 \sin (2\pi x/L),
    $$
    avec $T_0 = \SI{20}{\degree C}$. Remplir la première ligne du tableau \texttt{T} pour spécifier les conditions initiales de l'expérience.
\solution{
% \begin{minted}{python}
% T0 = 20
% T[0,:] = T0*np.sin(2*np.pi*x/L)
% \end{minted}
    }
    \item Compléter le code suivant pour tracer l'évolution de la température initiale au sein de la barre.
\begin{minted}{python}
plt.close()
plt.figure()
plt.grid()
plt.xlabel("x (m)")
plt.ylabel("T (°C)")
plt.title("Profil de température initial")
\end{minted}
    % \solution{
\begin{minted}{python}
plt.plot(x,T[0,:])
plt.show()
\end{minted}
    % }
    \end{questions}
    
    
    \begin{questions}
      \item	
      Remplir le tableau \texttt{T} de manière à imposer, à tout instant, les conditions aux limites aux deux extrémités de la barre, $T(x=0,t)=0$ et $T(x=L,t)=0$.
      % \solution{
        \begin{minted}{python}
        for i in range (1,Nt): # l'instant initial a déjà été traité
          T[i,0]  = 0
          T[i,-1] = 0
        \end{minted}
        % }
      \end{questions}
      
      \paragraph{Discrétisation de l'équation de diffusion}
      Il s'agit donc désormais de résoudre l'équation :
      $$\frac{\partial T}{\partial t} = \kappa \frac{\partial^2 T}{\partial x^2}.$$
      \begin{questions}
        \item En utilisant l'approximation de la dérivée, exprimer $\frac{\partial T}{\partial t}(t=i\mathrm{d}t,x=j\mathrm{d}x)$ en fonction des éléments de matrice \texttt{T[i,j]} et \texttt{T[i+1,j]}.
        \solution{
          $$
          \boxed{\dfrac{\partial T}{\partial t}(t=i\mathrm{d}t,x=j\mathrm{d}x) \simeq \frac{\texttt{T[i+1,j]}-\texttt{T[i,j]}}{\mathrm{d}t}}
          $$
        }
        \item Soit une fonction $f$. En écrivant la formule de Taylor à l’ordre 2 pour $f(x+\mathrm{d} x)$ et $f(x-\mathrm{d} x)$, démontrer que : 
        $$ f''(x)\simeq\frac{f(x+\mathrm{d} x)+f(x-\mathrm{d} x)-2f(x)}{\mathrm{d} x^2}.$$
        \solution{
Les développements de Taylor à l'ordre 2 de $f(x+\mathrm{d} x)$ et $f(x-\mathrm{d} x)$ sont respectivement : 
$$
f(x+\mathrm{d} x)\simeq f(x)+f'(x)\mathrm{d} x+\frac{f''(x)}{2}\mathrm{d} x^2
$$
et
$$
f(x-\mathrm{d} x)\simeq f(x)-f'(x)\mathrm{d} x+\frac{f''(x)}{2}\mathrm{d} x^2.
$$
En sommant ces deux expressions, on obtient :
$$
f(x+\mathrm{d} x)+f(x-\mathrm{d} x)\simeq 2f(x)+f''(x)\mathrm{d} x^2.
$$
En isolant $f''(x)$, on obtient bien la relation souhaitée.
        }
        \item En déduire l'approximation de la grandeur $\frac{\partial^2 T}{\partial x^2}(t=i\mathrm{d}t,x=j\mathrm{d}x)$ en fonction des éléments de matrice \texttt{T[i,j]}, \texttt{T[i,j+1]} et  \texttt{T[i,j-1]}.
        \solution{
          $$
          \boxed{\dfrac{\partial^2 T}{\partial x^2}(t=i\mathrm{d}t,x=j\mathrm{d}x) \simeq \frac{\texttt{T[i,j+1]}+\texttt{T[i,j-1]}-2\texttt{T[i,j]}}{\mathrm{d}x^2}.}
          $$
        }
        \item En déduire que la résolution de l'équation de diffusion thermique dans la barre se ramène au schéma numérique :
        \begin{center}
          $$ \boxed{
            \texttt{T[i+1,j]}=\texttt{T[i,j]}+\frac{\kappa \mathrm{d}t}{\mathrm{d}x^2} \left( \texttt{T[i,j+1]}+\texttt{T[i,j-1]}-2\texttt{T[i,j]}\right).
            }$$
        \end{center}
      \end{questions}
    
      \paragraph{Résolution}
      On peut montrer qu'un tel schéma numérique converge si $$\frac{\kappa \mathrm{d} t}{\mathrm{d} x^2}<\frac{1}{2}.$$ Avec les valeurs de $\mathrm{d} x$ et de $\mathrm{d} t$ choisies, cette condition est respectée pour la valeur choisie de la diffusivité thermique $\kappa=\SI{e-4}{m^2.s^{-1}}$.
      \begin{questions}
        \item	Implémenter ce schéma numérique en réalisant une boucle sur \texttt{i} et éventuellement sur \texttt{j} pour remplir le tableau \texttt{T} dans son intégralité.
        % \solution{
\begin{minted}{python}
kappa = 1e-4
for i in range(0,Nt-1): # on remplit à l'indice i+1
  for j in range(1,Nx-1): # les conditions aux limites sont déjà fixées
  T[i+1,j]=T[i,j]+kappa*dt/(dx**2)*(T[i,j+1]+T[i,j-1]-2*T[i,j])
\end{minted}
        % }
        \item On souhaite tracer le profil de température aux différents instants définis dans la liste ci-dessous. Compléter le code ci-dessous de sorte à tracer sur un même graphique légendé l'évolution du profil de température dans le temps.
    %     \begin{minted}{python}
    %     plt.close()
    %     plt.figure()
    %     plt.grid()
    %     plt.xlabel("x (m)")
    %     plt.ylabel("T (°C)")
    %     plt.title("Profil de température dans la barre à différents instants")
    
    %     instants = [0,1,3,6,9,12,15,30] # instants choisis en minutes
    % \end{minted}
    % % \solution{
    %   \begin{minted}{python}
    %   for t in instants: # tracé du profil du température aux instants choisis
    %     i = int(t*60/dt)
    %     plt.plot(x,T[i,:],label="t = "+str(t)+" min")
        
    % plt.legend()
    % plt.show()
    % \end{minted}
    % }
      \end{questions}
    \end{exo}
    %%%%%%%%%%%%%%%%%%%%%%%%%%%%%%%%%%%