% !TEX root = /home/eb/.vscode/qftrules.prepaworkshop/tmp/Exercice.tex
%%%%%%%%%%%%%%%%%%%%%%%%%%%%%%%%%%%%%%%%%%%%%%%%%%%%%%%%%%%%%%%%%%%%%%%%%
\begin{exo}[2][TD]{Analogie avec le champ gravitationnel}
Plusieurs résultats de l'électrostatique se généralisent à la gravitation. En effet, considérons deux masses ponctuelles, disons $m_O$ placée en $\rm O$, et $m$ placée en~$\rm M$. On note $\vv{OM}=r\,\vv{e_r}$ le vecteur position et $G=\SI{6,67e-11}{m^3.kg^{-1}.s^{-2}}$ la constante de gravitation.

\begin{questions}
\item Écrire la force gravitationnelle s’exerçant entre ces deux corps. Commenter la dépendance en $r$. Définir un champ gravitationnel $\vv{\mathcal{G}}(M)$ par analogie avec la champ électrostatique $\vv{E}(M)$. Compléter le tableau d’analogie suivant.
\begin{center}
\renewcommand{\arraystretch}{1.3}
\begin{tabular}{|l|l|l|l|l|}
\hline
\textsc{Électrostatique} & Champ $\vv{E}$ & Charge $q$ & Charge volumique $\rho$ & Permittivité du vide $\varepsilon_0$ \\
\hline \textsc{Gravitation} & & & &  \\
\hline
\end{tabular}
\end{center}


\solution{
    La force de gravitation newtonnienne a pour expression,
 $$\vv{F}_{m_O\rightarrow m}=\frac{-G m m_O}{r^2}\vv{e_r}=m\vv{\mathcal{G}}_{O}(M),$$
 avec le champ gravitationnel exercé par la masse en $\rm O$ au point $\rm M$, 
 $$
 \vv{\mathcal{G}}_{O}(M)=\frac{-G m_0}{r^2}\vv{e_r}.
 $$
 On peut donc dresser le tableau d'analogie suivant.
 \begin{center}
    \renewcommand{\arraystretch}{1.3}
    \begin{tabular}{|l|l|l|l|l|}
    \hline
    \textsc{Électrostatique} & Champ $\vv{E}$ & Charge $q$ & Charge volumique $\rho$ & Permittivité du vide $\varepsilon_0$ \\
    \hline \textsc{Gravitation} &  Champ $\vv{\mathcal{G}}$ & Masse $m$ & Masse volumique $\mu$ & $1/4\pi G$ \\
    \hline
    \end{tabular}
    \end{center}

}

\item Écrire deux équations locales de la gravitation qui portent sur $\vv{\mathcal{G}}(M)$. Définir un potentiel gravitationnel $\Phi(M)$ et déterminer l’équation vérifiée par $\Phi(M)$.
\solution{
Il est également possible d'écrire des équations locales pour le champ gravitationnel, sous la forme
 \begin{equation}
    \boxed{
 \begin{cases}
 {\rm div}{\,\vv{\mathcal{G}}}=-4\pi G \mu,\\
{ \vv{\rm rot}}{\vv{\mathcal{G}}}= 0.
 \end{cases}
    }
 \end{equation}
L’équation au rotationnel permet de définir  un potentiel gravitationnel $\Phi$ tel que 
$$\boxed{
    \vv{\mathcal{G}}=-{ \vv{\rm grad}\,}{\Phi}.
}
$$ Le potentiel gravitationnel vérifie alors l’équation de Poisson gravitationnelle, 
$$
\boxed{
\Delta \Phi = 4\pi G \mu.
}
$$
 }

\item Énoncer le \emph{théorème de Gauss gravitationnel}. Dans Star Trek apparaissent des sphères de Dyson, qui sont des sphères creuses homogènes de masse totale $M$ et de rayon $R$. Déterminer le champ gravitationnel $\vv{\mathcal{G}}(M)$ à l’intérieur et à l’extérieur d’une sphère de Dyson.

\solution{
    Comme la distribution de masse est invariante par rotation, $\vv{\mathcal{G}}(M)$ également et donc le champ gravitationnel est de la forme 
    $$\vv{\mathcal{G}}(M) = \mathcal{G}(r)\vv{u_r}.$$
    On applique le théorème de Gauss gravitationnel sur la surface fermée $\mathcal{S}= \{\text{sphère de rayon r}\}$. On a
\begin{equation}
\oiint_{S}\vv{\mathcal{G}}\cdot\vv{\mathrm{d} S} = \mathcal{G}(r) 4\pi r^2 = 4\pi G m_{\rm{int}},
\end{equation}
On obtient alors 
$$
\vec{\mathcal{G}}(r) = \frac{G m_{\rm int}}{r^2}.
$$

\begin{itemize}
\item Si \fbox{$r<R$}, la masse intérieur $m_{\rm int}=0$ est nulle, et donc
$$
\boxed{
    \vv{\mathcal{G}}(r<R) = \vv{0}.
}
$$
Les influences gravitationnelles de la sphère se compensent à l'intérieur !
\item Si \fbox{$r>R$}, la masse intérieure vaut $m_{\rm int}=M$, et donc
$$
\boxed{
\vv{\mathcal{G}}(r>R) = -\dfrac{\mathcal{G}M}{r^2}\vv{e_{r}}.
}
$$
Le champ à l'extérieur de la sphère est celui d'un corps ponctuelle à l'origine de même masse.
\end{itemize}
}

\item Quelles sont les limites de cette analogie ? (ordres de grandeurs, signe de la force, etc.)

\solution{ La différence principale entre le champ électrostatique et le champ gravitationnel est qu'il n'existe pas de masse négative dans le cas gravitationnel. Attention aussi aux ordres de grandeurs qui sont très différents: la force gravitationnelle agit principalement à très grande échelle (sur des objets très massifs). Aux petites et moyennes échelles, la force électromagnétique est dominante. Enfin, l’analogie entre électromagnétique et gravitation dans valable que dans la limite statique et non relativiste.
}

\end{questions}

\end{exo}
%%%%%%%%%%%%%%%%%%%%%%%%%%%%%%%%%%%%%%%%%%%%%%%%%%%%%%%%%%%%%%%%%%%%%%%%%
